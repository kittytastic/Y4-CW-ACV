We choose to train our patch model from scratch as opposed to improving the model in Q2.1, owing experience of starting from scratch in Q2.1.
Our results (see Figure \ref{fig:Q2_2}) were poor compared to the background model, which we attribute to multiple factors.
\begin{figure}[h!]
  \begin{center}
  \includegraphics[scale=0.2]{Q2_2_pad.jpg}
    \caption{Q2.2 sample of patch frames (left) and style transfer into other domain (right).}
    \label{fig:Q2_2}
  \end{center}
  \end{figure}

Firstly, our training time was approx $25\%$ of that of the background model.
Also, we didn't use any augmentations, we found that augmentations proved vital in Q2.1, so not using them was a significant disadvantage.

We did however learn that maintaining aspect ratio to standardize the size of patches was the best method, as mentioned in Q1.4.
There are a lot of places for improvement, many of which we apply to Q2.3.
It would also be interesting to see if adapting the background would be a better technique than training from scratch, as our final background models proved remarkably effective.
